\documentclass[12pt,a4paper]{paper}
\usepackage[utf8]{inputenc}
\usepackage[english]{babel}
\usepackage{amsmath}
\usepackage{enumitem}
\usepackage{amsfonts}
\usepackage{amssymb}
\usepackage[left=2cm,right=2cm,top=2cm,bottom=2cm]{geometry}
\usepackage{Sweave}
\begin{document}
\title{STAT646 - Homework 1\\\small{Daniel Osorio - dcosorioh@tamu.edu\\Department of Veterinary Integrative Biosciences\\Texas A\&M University}}
\maketitle
\Sconcordance{concordance:Osorio_Daniel_HW1.tex:Osorio_Daniel_HW1.Rnw:%
1 13 1 1 0 8 1 1 5 1 0 1 2 1 0 1 4 3 0 1 1 5 0 1 4 3 %
0 1 1 6 0 1 2 1 3 2 0 2 1 6 0 1 2 1 1 1 2 1 0 1 1 1 3 %
1 0 1 2 1 0 1 2 7 0 1 1 7 0 1 2 4 1 1 2 1 0 1 2 1 0 1 %
1 8 0 1 2 2 1 1 2 1 0 1 2 1 0 1 1 6 0 2 2 1 0 1 1 32 %
0 1 2 2 1 1 2 1 0 4 1 1 7 6 0 1 1 3 0 1 2 1 1 1 2 7 0 %
1 2 3 1 1 2 1 0 1 1 6 0 1 1 7 0 1 2 3 1}

\begin{enumerate}
\item Consider the human gene with HGNC symbol SPRR4.
\begin{enumerate}
\item Use the Biomart Ensembl database to obtain cDNA and peptide sequences for SPRR4.
What are the two sequences?
\begin{Schunk}
\begin{Sinput}
> library(biomaRt)
> mart <- useMart(biomart = "ensembl", 
+                 dataset = "hsapiens_gene_ensembl")
> peptideS <- getSequence(id="SPRR4", 
+                         type="hgnc_symbol", 
+                         seqType="peptide", 
+                         mart = mart)[,1]
> peptideS
\end{Sinput}
\begin{Soutput}
[1] "MSSQQQQRQQQQCPPQRAQQQQVKQPCQPPPVKCQETCAPKTKDPCAPQVKKQCPPKGTIIPAQ
     QKCPSAQQASKSKQK*"
\end{Soutput}
\begin{Sinput}
> cdnaS <- getSequence(id="SPRR4", 
+                      type="hgnc_symbol", 
+                      seqType="cdna", 
+                      mart = mart)[,1]
> cdnaS
\end{Sinput}
\begin{Soutput}
[1] "CTCTCCTGGGGTCCAGCTTGTCGCCTCTGGCTCACCTGTTCCTAGAGCAATGTCTTCCCAGCAG
     CAGCAGCGGCAGCAGCAGCAGTGCCCACCCCAGAGGGCCCAGCAGCAGCAAGTGAAGCAGCCTT
     GTCAGCCACCCCCTGTTAAATGTCAAGAGACATGTGCACCCAAAACCAAGGATCCATGTGCTCC
     CCAGGTCAAGAAGCAATGCCCACCGAAAGGCACCATCATTCCAGCCCAGCAGAAGTGTCCCTCA
     GCCCAGCAAGCCTCCAAGAGCAAACAGAAGTAAGGATGGACTGGATATTACCATCATCCACCAT
     CCTGGCTACCAGATGGAACCTTCTCTTCTTCCTTCTCCTCTTCCCTCCAGCTCTTGAGCCTACC
     CTCCTCTCACATCTCCTCCTGCCCAAGATGTAAGGAAGCATTGTAAGGATTTCTTCCCATCGTA
     CCCTTCCCCACACATACCACCTTGGCTTCTTCTATATCCCACCCCGATGCTCTCCCAGGTGGGT
     GTGAGAGAGACCTCATTCTCTGCAGGCTCCAGCGTGGCCACAGCTAAGGCCCATCCATTTCCCA
     AAGTGAGGAAAGTGTCTGGGCTTCTTCTGGGGTTCCACCCTGACAAGTAGGGTCACAGAGGCTG
     GTGCACAGTTTCTGCCTCATTCCTCTCCATGATGCCCCCTGCTCTGGGCTTCTCTCCTGTTTTC
     CCCAATAAATATGTGCCTCATGTAATAAA"
\end{Soutput}
\end{Schunk}
\item What is the Entrez ID for SPRR4?
\begin{Schunk}
\begin{Sinput}
> entrez <- getBM(attributes = c("hgnc_symbol", "entrezgene"), 
+                 mart = mart)
> entrezSPRR4 <- entrez[entrez[,1] %in% "SPRR4",2]
> entrezSPRR4
\end{Sinput}
\begin{Soutput}
[1] 163778
\end{Soutput}
\end{Schunk}
\item Retrieve GO information for SPRR4. What biological processes is the gene involved in?
Where in the cell is the SPRR4 protein located?
\begin{Schunk}
\begin{Sinput}
> library(org.Hs.eg.db)
> library(GO.db)
> GO <- mget(x = as.character(entrezSPRR4), 
+            envir = org.Hs.egGO)[[1]]
> GO <- data.frame(GO=unlist(lapply(GO, function(X){c(X[1])})), 
+                  ONTOLOGY =unlist(lapply(GO, function(X){c(X[3])})))
> as.vector(Definition(as.vector(GO[GO[,2] == "BP",1])))
\end{Sinput}
\begin{Soutput}
[1] "The formation of a covalent cross-link between or within protein 
     chains."                                                                                                                                    
[2] "The process in which a relatively unspecialized cell acquires 
     specialized features of a keratinocyte."                                                                                                       
[3] "The process in which the cytoplasm of the outermost cells of the 
     vertebrate epidermis is replaced by keratin. Keratinization occurs 
     in the stratum corneum, feathers, hair, claws, nails, hooves, 
     and horns."
\end{Soutput}
\begin{Sinput}
> as.vector(Term(as.vector(GO[GO[,2] == "CC",1])))
\end{Sinput}
\begin{Soutput}
[1] "cornified envelope" "cytoplasm"         
[3] "cell cortex"       
\end{Soutput}
\end{Schunk}
\end{enumerate}
\item Consider the human gene with HGNC symbol BRCA1.
\begin{enumerate}
\item Why is the BRCA1 gene relevant to breast cancer? \textit{`BRCA' is an abbreviation for `BReast CAncer gene.', it is a gene that encodes a nuclear phosphoprotein that plays a role in maintaining genomic stability, and it also acts as a tumor suppressor. Mutations in this gene are responsible for approximately 40\% of inherited breast cancers and more than 80\% of inherited breast and ovarian cancers.}
\item Which probeset on the Affymetrix HGU133a Gene Chip microarray corresponds to BRCA1?
\begin{Schunk}
\begin{Sinput}
> library("hgu133a.db")
> affyIds <- select(hgu133a.db, keys=keys(hgu133a.db), 
+                   columns = c("SYMBOL", "ENTREZID"))
> affyIds[affyIds[,2] %in% "BRCA1",]
\end{Sinput}
\begin{Soutput}
          PROBEID SYMBOL ENTREZID
4322  204531_s_at  BRCA1      672
12363 211851_x_at  BRCA1      672
\end{Soutput}
\end{Schunk}
\item According to the kegg Bioconductor package, what protein pathway is BRCA1 involved
in? Note: This is not the only pathway BRCA1 is involved in. The kegg package is not
complete here.
\begin{Schunk}
\begin{Sinput}
> library(KEGG.db)
> hsaPath <- unlist(mget(x=as.character(entrez[entrez[,1] %in% "BRCA1",2]), 
+             envir = KEGGEXTID2PATHID))
> KEGGPATHID2NAME[[gsub("hsa","",hsaPath)]]
\end{Sinput}
\begin{Soutput}
[1] "Ubiquitin mediated proteolysis"
\end{Soutput}
\end{Schunk}
\item What other genes are involved in the above protein pathway? Give their HGNC symbols
\begin{Schunk}
\begin{Sinput}
> pathGenes <- entrez[entrez[,2] %in% KEGGPATHID2EXTID[[hsaPath]],1]
> pathGenes
\end{Sinput}
\begin{Soutput}
  [1] "WWP2"    "PIAS2"   "PML"     "CDC26"   "FBXO4"  
  [6] "UBE2L3"  "CBLB"    "PPIL2"   "ANAPC13" "SKP1"   
 [11] "UBE2Q2"  "SMURF2"  "UBE2W"   "NEDD4"   "SYVN1"  
 [16] "UBE2R2"  "RNF7"    "TRIM37"  "HERC2"   "HERC4"  
 [21] "ELOC"    "UBE2G1"  "ANAPC11" "UBA6"    "CUL4B"  
 [26] "MGRN1"   "BTRC"    "MID1"    "DET1"    "DDB2"   
 [31] "RHOBTB2" "UBE2B"   "WWP1"    "UBE3A"   "UBE2C"  
 [36] "TRAF6"   "FANCL"   "VHL"     "STUB1"   "FBXW7"  
 [41] "SIAH1"   "CUL4A"   "PIAS4"   "UBE2L6"  "UBE2H"  
 [46] "ELOB"    "ANAPC1"  "XIAP"    "FZR1"    "AIRE"   
 [51] "HUWE1"   "UBE2O"   "UBE2U"   "CUL1"    "KLHL9"  
 [56] "BIRC6"   "UBE2E3"  "ITCH"    "UBA7"    "SMURF1" 
 [61] "UBE4A"   "UBR5"    "UBE2M"   "SOCS1"   "ERCC8"  
 [66] "DDB1"    "HERC1"   "CDC23"   "UBE2G2"  "UBE3C"  
 [71] "UBE2I"   "CUL2"    "NHLRC1"  "UBE2E2"  "CBLC"   
 [76] "ANAPC4"  "UBE2E1"  "UBA3"    "CUL5"    "PIAS3"  
 [81] "UBE2D1"  "UBE2NL"  "UBE2N"   "SAE1"    "NEDD4L" 
 [86] "ANAPC10" "CBL"     "FBXO2"   "CDC27"   "UBOX5"  
 [91] "UBE2K"   "KLHL13"  "CDC20"   "FBXW11"  "ANAPC2" 
 [96] "UBA1"    "UBA2"    "UBE2Z"   "UBE2F"   "UBE2S"  
[101] "CUL3"    "UBE2D3"  "UBE2D2"  "BIRC3"   "TRIM32" 
[106] "CUL7"    "BIRC2"   "SOCS3"   "MDM2"    "ANAPC7" 
[111] "KEAP1"   "UBE2D4"  "RBX1"    "RHOBTB1" "CDC34"  
[116] "PRPF19"  "CDC16"   "PRKN"    "UBE2J1"  "TRIP12" 
[121] "UBE2A"   "UBE2QL1" "PIAS1"   "MAP3K1"  "BRCA1"  
[126] "RCHY1"   "UBE3B"   "SKP2"    "COP1"    "FBXW8"  
[131] "UBE2Q1"  "UBE2J2"  "HERC3"   "ANAPC5"  "UBE4B"  
\end{Soutput}
\end{Schunk}
\item Use the topGO package to perform a GO enrichment analysis on the genes involved in
the above KEGG pathway. In the runTest function, use the “classic” algorithm and the
“fisher” test.
\begin{Schunk}
\begin{Sinput}
> library(topGO)
> allGenes <- rep(0,length(affyIds[,1]))
> names(allGenes) <- affyIds[,1]
> allGenes[affyIds[,2] %in% pathGenes] <- 1
> genSel <- function(X){return(X == 1)}
> GO_data <- new(Class = "topGOdata",
+                ontology = "BP",
+                allGenes = allGenes,
+                geneSel = genSel,
+                nodeSize = 10,
+                annot = annFUN.db,
+                affyLib = "hgu133a.db")
> enrichment <- runTest(GO_data, algorithm = "classic", 
                        statistic = "fisher")
\end{Sinput}
\end{Schunk}
\begin{enumerate}
\item How many GO terms have p-values < 0.001?
\begin{Schunk}
\begin{Sinput}
> sum(score(enrichment) < 0.01)
\end{Sinput}
\begin{Soutput}
[1] 944
\end{Soutput}
\end{Schunk}
\item What GO term has the smallest p-value (and is hence the “most enriched” in the
pathway genes)? Does it describe a cellular location, a biological process, or a
molecular function? How does the result compare to the KEGG pathway for BRCA1
that we found above?
\begin{Schunk}
\begin{Sinput}
> goTerm <- names(which.min(score(enrichment)))
> Definition(goTerm)
\end{Sinput}
\begin{Soutput}                                                                                                                                                                
"A protein modification process in which one or more groups of a small 
 protein, such as ubiquitin or a ubiquitin-like protein, are covalently 
 attached to a target protein." 
\end{Soutput}
\begin{Sinput}
> Term(goTerm)
\end{Sinput}
\begin{Soutput}
"protein modification by small protein conjugation" 
\end{Soutput}
\end{Schunk}
\end{enumerate}
\end{enumerate}
\end{enumerate}
\end{document}
